\begin{resumo}
Apresentamos um estudo sobre a aprendizagem de ciência de dados em cursos livres.
Mostramos que há uma diferença mensurável na aprendizagem de componentes relacionadas com ferramentas, quando comparada com a aprendizagem de componentes abstratas, como princípios e métodos.
Especulamos que essa diferença tenha origem na expectativa do aluno em conseguir um posto de trabalho, cujos requisitos relacionam-se explicitamente com as componentes concretas.
Argumentamos que esse fenômeno dificulta a adoção de currículos baseados em habilidades e competências e demonstramos que a satisfação do aluno com o curso, a relevância percebida por ele sobre os tópicos abordados e o ritmo sobre sua aprendizagem contribuem com no máximo aproximadamente 20\% da aprendizagem.
\end{resumo}

\begin{abstract}
%\vspace{-0.5cm}
This work presents a study about learning in data science courses.
We present a meagninful difference exists between the learning of concrete topics, usually related to tools, when compared to the learning of abstract topics, like methods and principles.
We speculate this difference is due to the students expectancy of getting a working position as a data scientist, whose requirements explicitly match concrete topics.
We argue this phenomenon hinders the adoption of competency-based curricula.
Finally, we show that students' satisfaction, perceived relevance of each topic and perceived rythm account for, at most, approximately 20\% of students learning.
\end{abstract}

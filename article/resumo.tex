\begin{resumo}
Apresentamos neste trabalho um estudo sobre como a expectativa de obter um posto de trabalho como cientista de dados afeta a aprendizagem do aluno em cursos livres de ciência de dados e promove uma situação de incompatibilidade com propostas de adoção de currículos baseados em competências.
Em vista disso propomos estratégias para contornar o problema.
Discutimos também a influência da satisfação do aluno com o curso, da relevância percebida por ele sobre os tópicos abordados e do rítmo sobre sua aprendizagem.
Mostramos que esses fatores explicam no máximo aproximadamente 20\% da aprendizagem.
\end{resumo}

\begin{abstract}
%\vspace{-0.5cm}
This work presents a study concerning how the expectancy, by the student, to achieve a work position affects his learning on courses of data science, and how this effect promotes incompatibility with proposals of adoption of competency-based curricula for these courses.
Based on that, we propose solutions to this problem.
We also argue about the influence of satisfaction with the course, perceived relevance about the topics covered by the course and the pace of his learning.
We show these factors explain, at most, approximately 20\% of student's learning.
\end{abstract}
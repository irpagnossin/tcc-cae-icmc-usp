\section{Referencial teórico}
\subsection{Aprendizagem baseada em competências}
\label{sec:competencias}

Uma competência é uma coleção de habilidades e conhecimentos para realizar uma tarefa \cite{Voorhees2001} ou, segundo a BNCC, ``a mobilização de conhecimentos (conceitos e procedimentos), habilidades (práticas, cognitivas e socioemocionais), atitudes e valores para resolver demandas complexas da vida cotidiana, do pleno exercício da cidadania e do mundo do trabalho''.

A aprendizagem baseada em competências é aquela que define objetivos de aprendizagem em termos dessas competências, que devem ser mensuráveis.
Isto é, ``se uma proposta de competência não puder ser descrita sem ambiguidade ou ser medida, provavelmente não é uma competência'' \cite{Voorhees2001}.

No Brasil a adoção dessa metodologia no Ensino Básico foi oficializada por meio da BNCC, aprovada em 2017, que define os conhecimentos, competências e habilidades mínimos que todos os cidadãos brasileiros devem desenvolver antes do Ensino Superior.

Embora a BNCC limite-se à Educação Básica, existe a preocupação de utilizar também essa abordagem no Ensino Superior, particularmente em resposta ao mercado de trabalho.
De fato, a Unesco já defendia as bases da Educação para o século XXI com seus pilares: (1) aprender a conhecer, (2) aprender a fazer, (3) aprender a conviver e (4) aprender a ser.

Realmente, esse modelo surgiu nos Estados Unidos, nos anos 1960, com o intuito de resolver a crescente disparidade entre as necessidades do mercado de trabalho e o que as universidades oferecem aos seus estudantes \cite{Zulauf2006}:

\begin{mycitation}
	``[No] mercado e os ambientes de trabalho, o ensino superior vem sofrendo crescente pressão para desenvolver a empregabilidade dos estudantes e tornar-se mais relevante no que diz respeito às necessidades dos empregadores.''
\end{mycitation}

De fato, segundo \cite{Gonczi1999} (Google Books), ``\foreign{What many of the reforms have in common is that the content of VEC courses as well as workplace training and assessment is based on occupational competency standards}''.

Embora haja inúmeras iniciativas \foreign{ad hoc} de criar um currículo para o ensino da ciência de dados \cite{Hassan2019, Anderson2014, Cheng2019} e iniciativas de unificação estejam surgindo agora \cite{Raj2019}, uma das que merece menção é a \foreign{EDISON Data Science Framework}, desenvolvida pela \foreign{International Association of Business Analytics Certification} (IABAC) e que provê uma base para a definição da profissão de cientista de dados, bem como componentes relacionados como educação, treinamento, papéis, dentre outros.
Ela define três especificações:
\begin{compactitem}
	\item \foreign{Data Science Competence Framework} (CF-DS) é o núcleo da especificação, que inclui as competências necessárias para o cientista de dados atuar no mercado de trabalho e na academia ao longo de toda sua carreira.
	\item \foreign{Data Science Body of Knowledge} (DS-BoK) define áreas de conhecimento para a construção de currículos de ciência de dados que comportem as competências identificadas no CF-DS \cite{Demchenko2017}.
	\item \foreign{Data Science Model Curriculum} (MC-DS) define objetivos de aprendizagem consonantes com a CF-DS e unidades de aprendizagem associadas às unidades de conhecimento definidas no DS-BoK.
\end{compactitem}

\subsection{Teoria do valor da expectativa}
%\cite{Wigfield2000}
%\cite{Guo2015}?

Em 1964, Victor H. Vroom desenvolveu a sua teoria comportamentalista do valor da expectativa\footnote{\url{iedunote.com/expentancey-theory}.}, uma teoria da motivação, segundo a qual ``a escolha, persistência e desempenho de indivíduos pode ser explicada por sua crença sobre quão bem ele executará uma atividade e quanto ele valoriza essa atividade''.
A teoria propõe ainda que a motivação depende de três fatores: 
\begin{compactitem}
	\item Resultado ou recompensa esperado, chamado de valência;
	\item Percepção de \emph{intensidade} da relação entre o desempenho requerido e o resultado (instrumentalidade);
	\item Percepção do \emph{vínculo} existente entre o esforço requerido e o desempenho subsequente (expectativa).
\end{compactitem}

\subsection{Aprendizagem significativa}

Aprendizagem significativa\footnote{\url{portaldoprofessor.mec.gov.br/storage/materiais/0000012381.pdf}} é a concepção cognitivista de ensino e aprendizagem proposta pelo psicólogo americano David Ausubel em 1963.
Ela afirma que o fator isolado mais relevante para a aprendizagem é o conhecimento prévio do aluno.
Ou seja, a aprendizagem ocorre quando novas informações ancoram-se em conceitos ou proposições relevantes pré-existentes.


\section{Referencial teórico}
\label{sec:referencial-teorico}

Nesta seção apresentamos as referências teóricas que suportam o desenvolvimento deste trabalho: começamos abordando a aprendizagem baseada em competências ({Seção~\ref{sec:competencias}}), com o intuito de justificar sua escolha como base para propostas de cursos de ciência de dados.
Em seguida, tratamos da teoria do valor da expectativa (Seção~\ref{sec:tve}), que utilizamos para conjecturar a razão pela qual observamos os resultados deste trabalho.
Finalmente, mencionamos a aprendizagem significativa (Seção~\ref{sec:as}) pois, conforme nossas conclusões, pode ser utilizada para intervir num dos resultados observados.

\subsection{Aprendizagem baseada em competências}
\label{sec:competencias}

Uma competência é uma coleção de habilidades e conhecimentos para realizar uma tarefa.
Alternativamente, a BNCC \cite{BNCC} define competência como ``a mobilização de conhecimentos (conceitos e procedimentos), habilidades (práticas, cognitivas e socioemocionais), atitudes e valores para resolver demandas complexas da vida cotidiana, do pleno exercício da cidadania e do mundo do trabalho''.

A aprendizagem baseada em competências é aquela que define objetivos de aprendizagem em termos dessas competências, que devem ser mensuráveis.
Realmente, a ``educação baseada em competências atualmente foca em resultados de aprendizagem e aborda o que os alunos devem aprender a \emph{fazer}'' \cite{Butova2015} (tradução e grifo nossos).

Segundo \cite{Butova2015}, a ideia original surgiu em 1965 com o filósofo e linguista americano Noam Chomsky, que enfatizou, no contexto linguístico, a diferença entre conhecer um idioma e saber aplicá-lo.
Mais tarde essa diferença foi extrapolada para outras áreas, como a pedagogia e a filosofia.

No Brasil a adoção dessa metodologia no Ensino Básico foi oficializada por meio da BNCC, homologada em 2018, que define os conhecimentos, competências e habilidades mínimos que todos os cidadãos brasileiros devem desenvolver antes do Ensino Superior.

Embora a BNCC limite-se à Educação Básica, existe a preocupação de utilizar também essa abordagem no Ensino Superior, particularmente em resposta ao mercado de trabalho.
Mais do que isso, a Organização das Nações Unidas para a Educação, a Ciência e a Cultura (Unesco) defende que a Educação para o século XXI deve desenvolver ``competências globais''; os quatro pilares da Educação moderna: (1) aprender a conhecer, (2) aprender a fazer, (3) aprender a conviver e (4) aprender a ser.

De fato, a Educação baseada em competências é recorrentemente citada como um meio para resolver a crescente disparidade entre as necessidades do mercado de trabalho e o que as universidades oferecem aos seus estudantes \cite{Zulauf2006}: 

\begin{mycitation}
	``[No] mercado e os ambientes de trabalho, o ensino superior vem sofrendo crescente pressão para desenvolver a empregabilidade dos estudantes e tornar-se mais relevante no que diz respeito às necessidades dos empregadores.''
\end{mycitation}

Concomitantemente a esse movimento, inúmeras iniciativas \foreign{ad hoc} de criar um currículo para o ensino da ciência de dados, tais como \cite{Hassan2019}, \cite{Anderson2014} e \cite{Cheng2019} tem surgido; e começam a aparecer iniciativas de unificação \cite{Raj2019}.

Uma das que merece menção é a \foreign{EDISON Data Science Framework}, desenvolvida pela IABAC, que provê uma base para a definição da profissão de cientista de dados, bem como componentes relacionados como educação, treinamento, papéis, dentre outros.
Ela define três especificações:
\begin{compactitem}
	\item \foreign{Data Science Competence Framework} (CF-DS) é o núcleo da especificação, que inclui as competências necessárias para o cientista de dados atuar no mercado de trabalho e na academia ao longo de toda sua carreira.
	\item \foreign{Data Science Body of Knowledge} (DS-BoK) define áreas de conhecimento para a construção de currículos de ciência de dados que comportem as competências identificadas no CF-DS \cite{Demchenko2017}.
	\item \foreign{Data Science Model Curriculum} (MC-DS) define objetivos de aprendizagem consonantes com a CF-DS e unidades de aprendizagem associadas às unidades de conhecimento definidas no DS-BoK.
\end{compactitem}

Assim, defendemos que os currículos de cursos de ciência de dados devam ser baseados em competências e habilidades.
Porém, o mercado impõe dificuldade a essa empreitada pois, ao enfatizar ferramentas nas vagas de emprego, incentiva o candidato a cientista de dados a procurar cursos que lhe deem esse \emph{conhecimento}.
Procuramos fundamentar essa suposição com base na teoria do valor da expectativa, na próxima seção.

\subsection{Teoria do valor da expectativa}\label{sec:tve}
%\cite{Wigfield2000}
%\cite{Guo2015}?

% iedunote.com/expentancey-theory
Em 1964, Victor H. Vroom desenvolveu a sua teoria comportamentalista do valor da expectativa \cite{Petri}, uma teoria da motivação, segundo a qual ``a escolha, persistência e desempenho de indivíduos pode ser explicada por sua crença sobre quão bem ele executará uma atividade e quanto ele valoriza essa atividade''.
A teoria propõe ainda que a motivação depende de três fatores: 
\begin{compactitem}
	\item Resultado ou recompensa esperado, chamado de valência;
	\item Percepção de intensidade da relação entre o desempenho requerido e o resultado (instrumentalidade);
	\item Percepção do vínculo existente entre o esforço requerido e o desempenho subsequente (expectativa).
\end{compactitem}

Conforme demonstraremos nos resultados deste trabalho (Seção~\ref{sec:resultados}), a aprendizagem percebida pelo aluno em aulas explicitamente relacionadas com ferramentas é maior, dentro de um nível de significância de 95\%, que aquela em outras aulas.
Aplicamos a teoria do valor da expectativa para conjecturar que essa diferença é devida a uma maior intensidade do vínculo entre as aulas de ferramentas e as vagas de emprego (valência), levando a motivação intrínseca que promove a aprendizagem nessas aulas.

Além disso, supondo verdadeira essa conjectura, podemos ainda utilizá-la para promover a aprendizagem nas demais aulas.
Para isso, propomos (1) intensificar, pelo discurso, o vínculo entre os requisitos do mercado e os objetivos das aulas, e/ou (2) desenvolver essas aulas utilizando ferramentas, se possível, o que se baseia na aprendizagem significativa, objeto da próxima seção.

\subsection{Aprendizagem significativa}\label{sec:as}

% portaldoprofessor.mec.gov.br/storage/materiais/0000012381.pdf
Aprendizagem significativa \cite{Pelizzari2002} é a concepção cognitivista de ensino e aprendizagem proposta pelo psicólogo americano David Ausubel em 1963.
O autor afirma que o fator isolado mais relevante para a aprendizagem é o conhecimento prévio do aluno.
Ou seja, a aprendizagem ocorre quando novas informações ancoram-se em conceitos ou proposições relevantes pré-existentes.

Desse modo, podemos ancorar a aprendizagem de componentes abstratas, isto é, que têm um vínculo fraco com a valência, na aprendizagem das componentes cujo vínculo é forte (\ie, nos quais verificamos maior aprendizagem percebida).


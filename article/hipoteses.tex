\subsection{Hipóteses}
\label{sec:hipoteses}

As hipóteses propostas para este trabalho foram:
\begin{compactitem}
	\item \textbf{Hipótese 1:} a aprendizagem dos alunos de DA e DS é maior nas aulas que abordam explicitamente as ferramentas (\eg, Google Analytics para DA e Python para DS), quando comparadas às aulas mais abstratas, sobre princípios, técnicas e métodos (\eg, arquitetura de dados para DA e princípio de funcionamento dos algoritmos de agrupamento para DS).

	\item \textbf{Hipótese 2:} a aprendizagem dos alunos de DS (não de DA) é maior nas aulas que abordam algoritmos explicitamente (\eg, ``MeanShift e DBSCAN'') do que naquelas com tópicos mais abstratos (\eg, ``o funcionamento de um neurônio'').

	\item \textbf{Hipótese 3:} a relevância de cada tópico, o ritmo da aula e satisfação do aluno com o curso num dado instante de tempo são preditores que também afetam a aprendizagem.
\end{compactitem}

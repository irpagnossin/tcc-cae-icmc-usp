\section{Hipóteses}
\label{sec:hipóteses}

As hipóteses propostas para este trabalho foram:
\begin{compactdesc}
	\item[Hipótese 1:] a aprendizagem dos alunos de DA e DS é maior nas aulas que abordam explicitamente as ferramentas (\eg, Google Analytics para DA e Python para DS), quando comparadas às aulas mais abstratas, sobre princípios, técnicas e métodos (\eg, arquitetura de dados para DA e princípio de funcionamento dos algoritmos de agrupamento para DS).

	A justificativa para essa hipótese é que as aulas sobre ferramentas guardam uma relação explícita com os requisitos das vagas de trabalho, almejadas pelos alunos.
	Cono consequência, conforme a teoria do valor da expectativa, a expectativa de que dominar esse tópico possa levar ao resultado desejado (emprego) resulta em motivação intrínseca que contribui para a aprendizagem.
	Porém, observe que essa justificativa não faz parte da hipótese proposta, haja vista que não podemos testá-la com os dados em mãos.

	\item[Hipótese 2:] a aprendizagem dos alunos de DS (não de DA) é maior nas aulas que abordam algoritmos explicitamente (\eg, ``MeanShift e DBSCAN'') do que naquelas com tópicos mais abstratos (\eg, ``o funcionamento de um neurônio'').
	A justificativa é análoga à da hipótese anterior.

	\item[Hipótese 3:] a relevância de cada tópico, o ritmo da aula e satisfação do aluno com o curso num dado instante de tempo são preditores que também afetam a aprendizagem.

	Nesse caso a justificativa baseia-se parcialmente na aprendizagem individualizada (\foreign{self-paced learning}), na qual o rítmo da aprendizagem é completamente estabelecido pelo aluno.
	Além disso, há motivações pragmáticas: a relevância, o ritmo e a satisfação são parâmetros que podem ser modificados.
	Por exemplo, a relevância pode ser abordada com base na aprendizagem significativa; o rítmo, com melhores planejamentos de aula e do curso.
	A satisfação é o parâmetro mais abstrato e abrangente, que envolve inclusive fatores institucionais como a infraestrutura, suporte ao aluno \etc.
\end{compactdesc}

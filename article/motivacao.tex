\section{Cenário e motivação}\label{sec:motivacao}

Em vista do exposto, o autor deste trabalho, que atua como coordenador na instituição educacional Digital House (\url{digitalhouse.com/br}), que por sua vez oferece cursos livres de DS e \foreign{Data Analytics} (DA), gostaria de adotar a aprendizagem baseada em competências nos cursos de ciência de dados e de \foreign{Data Analytics} (DA) sob sua responsabilidade.

Neste trabalho nós analisamos os resultados de aprendizagem nos cursos de DA e DS oferecidos pela instituição educacional mencionada, com o intuito de identificar uma possível concorrência entre os currículos baseados em competências e os incentivos do mercado de trabalho, conforme exposto nas hipóteses deste trabalho (Seção~\ref{sec:hipoteses}).

Conforme a definição de ciência de dados do NIST, os cursos de DA e DS podem ser classificados como de ciência de dados.
Porém, eles guardam semelhanças e diferenças entre si:
\begin{compactitem}
	\item \textbf{\foreign{Data Analytics} (DA):} visa a inteligência de mercado (\foreign{Business Intelligence}, BI), isto é, a aplicação da ciência de dados para obter conhecimento acionável que suporte decisões estratégicas para um empreendimento.
	Esse curso tem carga horária de 140 horas e dura 14 semanas.
	Outra característica desse curso é que ele baseia-se em aplicativos como PowerBI, Tableau, MySQL \etc.

	\item \textbf{\foreign{Data Science} (DS):} visa o desenvolvimento de ``produtos de dados'', isto é, \foreign{softwares} que automaticamente obtém conhecimento acionável para oferecê-lo aos clientes.
	Esse curso tem carga horária de 196 horas, dura 19 semanas e baseia-se na linguagem de programação Python e suas extensões para análise de dados.
\end{compactitem}

A Tabela~\ref{tab:da-vs-ds} resume o que foi exposto imediatamente acima, comparando os dois cursos.
Note que, atualmente, os cursos em questão \emph{não} são baseados em competências.
De fato, esse é nosso objetivo geral.

\begin{table}
	\caption{Comparação dos cursos de DA e DS.}
	\label{tab:da-vs-ds}
	\footnotesize
	\begin{tabular}{lllll}
		\toprule
		Curso & Objetivo & \shortstack[l]{Carga\\horária (horas)} & \shortstack[c]{Duração\\(semanas)} & Ferramenta\\
		\midrule
		DA & Inteligência de mercado & 140 & 14 & Aplicativos (\eg, PowerBI)\\
		DS & Produto de dados & 196 & 19 & Linguagem de programação (\eg, Python)	\\
		\bottomrule
	\end{tabular}
\end{table}
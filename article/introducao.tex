\section{Introdução}

O interesse pela ciência de dados tem crescido nos últimos anos, assim como a quantidade de vagas de trabalho nessa área.
Consequentemente, a demanda e a oferta por cursos nessa área seguem a mesma tendência.
Porém, a definição de um currículo global de ciência de dados ainda não existe, haja vista a interdisciplinaridade dessa área.

Concomitantemente, a adoção da Educação baseada em competências como meio para desenvolver os quatro pilares da Educação do século XXI (aprender a conhecer, aprender a fazer, aprender a conviver e aprender a ser) é também um fenômeno global.
Nesse contexto, a \foreign{International Association of Business Analytics Certification} (IABAC) têm construído um arcabouço unificado e globalizado de conhecimentos, habilidades e competências para a atuação do cientista de dados.

Apesar disso, o discurso das empresas que procuram cientistas de dados no Brasil guarda uma relação explícita com conhecimentos apenas, em especial sobre ferramentas e algoritmos, sugerindo um distanciamento de qualquer proposta baseada em competências.
Isso pode dificultar, especialmente em cursos livres, o foco nas competências, pois o candidato a cientista de dados, ao confrontar os requisitos dessas vagas com as ofertas de cursos, vê mais claramente a relação com aqueles cursos que focam sua comunicação (propaganda) nesses conhecimentos.

Estamos particularmente interessados no contexto dos cursos livres, pois é nele que o primeiro autor está inserido profissionalmente.
Assim, tomamos como problema de pesquisa a adoção da aprendizagem baseada em competências em cursos livres de ciência de dados.
O objetivo geral é viabilizar essa empreitada e, para isso, começamos propondo um objetivo específico: averiguar se o fenômeno explicado no parágrafo anterior de fato existe.

Para isso, propusemos três hipóteses: a primeira delas afirma que a aprendizagem dos alunos em componentes explicitamente relacionadas com ferramentas é maior do que em componentes mais abstratas.
A segunda hipótese é similar, mas substitui ``ferramentas'' por ``algoritmos''.

A ideia subjacente a essas hipóteses é que as componentes abstratas, que são igualmente importantes para a composição de habilidades, são menos relevantes aos olhos dos alunos devido à ênfase que as ferramentas e algoritmos têm em anúncios de vagas de trabalho.

Finalmente, a terceira hipótese afirma que podemos utilizar a relevância de cada componente, conforme percebebida pelo aluno, o ritmo das aulas e a satisfação geral do aluno com o curso como preditores da aprendizagem.
Ou seja, podemos prever a aprendizagem com base nesses parâmetros.
A ideia aqui é que, se ao menos uma das duas primeiras hipóteses for verdadeira, podemos esperar alguma correlação com a relevância de cada componente, conforme percebida pelo aluno, e avaliar sua importância em comparação com o ritmo do curso e a satisfação geral do aluno.
Isso porque, mesmo que haja o efeito esperado, sua magnitude pode ser desprezível quando comparada com outros fatores.

Nós testamos essas hipóteses por meio de análise quantitativa sobre as respostas a pesquisas de levantamento apresentados aos alunos no final de cada aula.
Começamos aprofundando o cenário apresentado nos parágrafos anteriores: primeiramente, abordamos a evolução da ciência de dados e da demanda por postos de trabalho e cursos nessa área (Seção~\ref{sec:ds}).
Em seguida, voltamos nossa atenção para Educação baseada em competências e sua influência em propostas globalizadas de currículos de ciência de dados (Seção~\ref{sec:hc}).
Depois, apresentamos detalhes do cenário de aplicação deste trabalho e a motivação para ele (Seção~\ref{sec:motivacao}).
Seguimos com as referências teóricas (Seção~\ref{sec:referencial-teorico}), detalhes da metodologia de desenvolvimento deste trabalho e anállise (Seção~\ref{sec:metodologia}) para, então, apresentarmos e discutirmos os resultados (Seção~\ref{sec:resultados}).
Concluimos (Seção~\ref{sec:conclusao}) resumindo os resultados, enfatizando as limitações a apresentando possíveis prosseguimentos.
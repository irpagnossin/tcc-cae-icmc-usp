\section{Conclusão}
\label{sec:conclusao}

Neste trabalho nós avaliamos a influência da expectativa do aluno (de obter um posto de trabalho como cientista de dados) na sua aprendizagem e como isso pode ser incongruente com propostas de adotar currículos baseados em competências e habilidades em cursos livres de ciência de dados.
Avaliamos ainda como a satisfação do aluno com o curso, a relevância percebida por ele sobre os tópicos abordados e o ritmo do curso também infererem na aprendizagem.

Concluimos que nas aulas de ferramentas de ciências de dados os alunos apresentam maior aprendizagem, possivelmente devido à relação explícita delas com os requisitos de vagas de trabalho.
Esse resultado sugere duas estratégias para aprimorar a aprendizagem nas demais aulas: (1) desenvolvê-las utilizando as ferramentas (aprendizagem significativa) e (2) tornando mais evidente a relação dessas aulas com os requisitos do mercado (teoria do valor da expectativa).
Concluimos ainda que efeito análogo \emph{não} acontece para as aulas de algoritmos.

Finalmente, demonstramos o ritmo do curso, a relevância dos tópicos abordados e a satisfação com o curso são capazes de explicar no máximo 20\% da aprendizagem.
Esse resultado é inconclusivo, mas sugere um prosseguimento: decompor a satisfação nas suas componentes, propondo novos experimentos com os alunos.
